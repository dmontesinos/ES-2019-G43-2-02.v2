\section{Agenda}

\begin{itemize}
\subsection{Sprint 1}
\item Bienvenida: Inicio de la sesión de presentaciones.
\item Presentación: Exposición de cada grupo de la tarea realizada durante el Sprint 1.
\item Organización: Organización de las tareas a realizar durante el Sprint 2.
\begin{enumerate}
\item Documento MoM: Creación del documento
\item Acabar GameDoc: Faltaban algunas partes del documento
\item Presentación Sprint 2 : Creación de la presentación
\item Buscar información: Información referente a Bitbucket y su funcionamiento
\item Trazabilidad: Añadirla al documento correspondiente
\item Diagrama Casos de uso : Terminar el diagrama
\item Diagrama de clase: Crear el diagrama de clases
\item Correcciones documentos: Corregir versionado, nombres, etc.
\item Reasignar tipo de Prioridad y Verification Parents : Reasignar en documento de requerimientos
\item Crear y documentar documento UML clases: Crearlo mediante una plantilla anterior
\item Ampliar y terminar requerimientos más específicos
\end{enumerate}



\subsection{Sprint 2}
\item Presentaciones: Exposición de cada grupo de la tarea realizada durante el Sprint 2.
\item Organización: Organización de las tareas a realizar durante el Sprint 3.
\begin{enumerate}
	\item Documentar MoM v2.0: Documentar acta de la reunión.
	\item Diagrama de secuencia: Crear el diagrama de secuencia.
	\item Documentar diagrama de secuencia: Documentación del diagrama de secuencia.
	\item Diagrama de estados: Crear el diagrama de estados.
	\item Documentar diagrama de estados: Documentación del diagrama de estados.
	\item Diagrama de clase: Añadir cardinalidad al diagrama de clase.
	\item Documento GameDoc: Arreglar la versión.
	\item Documento GameDoc: Hacer apartado 2.
	\item Documento Requerimientos: Comprovar trazabilidad.
	\item Documento Requerimientos: Asignación correcta de Test o Diseño.
	\item Documento Requerimientos: Refinar los requerimientos.
	\item Presentación: Crear la presentación del Sprint 3.
	\item Revisar: Revisión de los documentos creados i/o modificados durante el Sprint (Especificaciones, Clases, GameDoc, Requerimientos, Secuencia, Estados)
	\item Hacer commits de las modificaciones en bitbucket.
\end{enumerate}



\subsection{Sprint 3}
\item Presentaciones: Exposición de cada grupo de la tarea realizada durante el Sprint 2.
\item Organización: Organización de las tareas a realizar durante el Sprint 4.
\begin{enumerate}
	\item Documento MoM v3.0: Documentar acta de la reunión.
	\item Presentación: Crear ala presentación del Sprint 4.
	\item Diagrama de clase:	 Arreglar errores:
	\begin{enumerate}
		\item Añadir nombre a las relaciones.
		\item Cambiar aquellas funciones necesarias de Jugador a Partida.
		\item Revisar que estén todas la funciones del diagrama de secuencia generado.
		\item Añadir funciones a la clase Interfaz.
		\item Conectar la clase Control a Partida y añadir las funciones a la clase Control.
	\end{enumerate}
	\item Diagrama de estado: Generar diagrama de estado, ya que la primera versión no era correcta.
	\item Documentar diagrama de estados: Explicar en que consiste el diagrama generado.
	\item Requerimientos:
	\begin{enumerate}
		\item Asignar los requerimientos correctamente en función si son de Test y/o Diseño.
		\item Refinar los requerimientos.
	\end{enumerate}
	\item Diagrama de secuencia: Revisar el diagrama y cambiar aquellas entidades BD por la correcta.
	\item Documentar diagrama de secuencia: Explicar en que consiste el diagrama de secuencia generado.
	\item Casos de test: Generar los casos de test de nuestro proyecto.
	\item Documentar Casos de Test: Explicar en que consiste cada caso de test generado.
\end{enumerate}



\end{itemize}
