\section{Historias}

Se han recogido todas las necesidades para llevar a cabo el desarrollo.
\begin{itemize}
	\item El usuario se conectará al sistema mediante un usuario y una contraseña.
	\item Para pasar de nivel los usuarios deberán completar los objetivos del nivel.
	\item El juego tendrá un ranking que se podrá consultar desde internet.
	\item En las habitaciones se podrán encontrar objetivos “bonus” que ayudarán al usuario a pasarse los niveles o acceder a los siguientes.
	\item En algunos niveles aparecerán pruebas contrarreloj que obligarán al usuario a pasarlo en tiempo o serán eliminados automáticamente.
	\item Si el jugador es eliminado, deberá volver a empezar la partida desde el principio.
	\item El juego tendrá los roles de administrador y de jugador normal.
	\item El administrador podrá crear y editar nuevos mapas.
	\item El administrador podrá modificar la contraseña del usuario.
	\item El administrador podrá eliminar la cuenta del usuario.
	\item El administrador podrá modificar el ranking de los usuarios.
	\item El usuario puede modificar o eliminar sus datos de la cuenta.
	\item El sistema se encargará de gestionar todo lo relacionado con el juego (rankings, cuentas, etc).
	\item El sistema será el encargado de generar los mapas aleatoriamente y sus propiedades (habitaciones, pruebas contrarreloj, etc).
	\item Habrá cierta cantidad de mapas de cada dificultad, que serán seleccionados por el sistema de forma aleatoria.
	\item El sistema gestionará los niveles de menor a mayor dificultad.
	\item El administrador podrá ver el estado de los usuarios online (jugando, esperando, etc).
	\item El administrador podrá crear cuentas nuevas.
	\item El jugador puede continuar la partida por donde la dejó.
	\item El sistema se encargará de guardar el estado de la partida del usuario.
	\item El jugador podrá empezar una partida nueva.
	\item El jugador podrá borrar una partida existente.
	\item En la pantalla se mostrarán los bonus obtenidos anteriormente.
	\item En la pantalla se mostrará si el jugador está en una pantalla contrarreloj.
	\item En cada turno del jugador, este podrá ver una pantalla con las posibles acciones que puede realizar (menú de movimientos).
	\item El jugador podrá guardar la partida manualmente.
	\item El jugador tendrá en el menú la opción para salir del juego.
	\item El jugador tendrá en el menú la opción para volver al menú principal.
	\item El jugador tendrá en el menú la opción para modificar sus ajustes.
	\item El jugador tendrá en el menú la opción para ver los rankings.
	\item El jugador tendrá en el menú la opción para empezar una partida nueva.
	\item El jugador tendrá en el menú la opción para gestionar sus partidas existentes.
\end{itemize}
