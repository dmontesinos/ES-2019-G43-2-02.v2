% @author Lluis Gesa (lluis.gesa@uab.cat)
%
% Enginyeria Del Software
% -----------------------------------------------------------------------------------



% definim quin estil general volem per el document
\documentclass[11pt]{article}

% indiquem quins paquets amb funcionalitat extra volem fer servir
\usepackage{a4wide, amsmath, tabularx, colortbl, fancyhdr, graphicx, lastpage, caption}
\usepackage[utf8]{inputenc}

% Indiquem que comencem a definir el document i els seus continguts.
\begin{document}

% Definim valors i propietats, creem macros que ens ajudaran a
% agilitzar coses repetides:

\newcommand{\NomProjecte}{Perdidos en la ignorancia}
\newcommand{\NomDocument}{Requerimentos}
\newcommand{\IdentificadorDocument}{GG-SL-N-REQ}
\newcommand{\VersioDocument}{v4.0}

% Definim una macro que cada cop que insertem \requeriment, creara una
% taula amb el contingut passat per paràmetre
\newcommand{\requeriment}[6]{
\begin{center}
	\begin{tabularx}{\linewidth}{|p{2.5cm}|X|}
	\hline
	\cellcolor[gray]{.8} \textbf{ID} & \cellcolor[gray]{.9} #1\\
	\hline
	\cellcolor[gray]{.8} \textbf{Title} & #2 \\
	\hline
	\cellcolor[gray]{.8} \textbf{Description} & #3\\
	\hline
	\cellcolor[gray]{.8} \textbf{Priority} & #4 \\
	\hline
	\cellcolor[gray]{.8} \textbf{Verification} & #5 \\
	\hline
	\cellcolor[gray]{.8} \textbf{Parents} & #6\\
	\hline
	\end{tabularx}
\end{center}
}

% Fent servir propietas del paquet fancyhdr per crear una capçalera
% que es mostrarà cada pàgina
\pagestyle{fancy}
\fancyhead[LO,LE]{\includegraphics[width = 3.5cm]{./imatges/uab-logo.png}}
\fancyhead[CO,CE]{\bf \NomProjecte \\ \vspace{\baselineskip}
  \NomDocument }
\fancyhead[RO,RE]{
{\bf Ref : \IdentificadorDocument} \\
Versió: \VersioDocument\\
Data: \today \\
Pag. \thepage ~of \pageref{LastPage}
}
\fancyfoot[LO,LE,CO,CE,RO,RE]{}
\textheight=19cm


%
% Creem la pàgina inicial :

% Insertem el nom del document en una taula centrada feng servir una
% font gran \huge

\begin{center}
\begin{tabular*}{15cm}{p{14.5cm}}
\\
\\
\\
\rowcolor[gray]{.9}\begin{center}\huge{\bf{\NomDocument}}\end{center}\\
\\
\end{tabular*}
\end{center}


%
% Insertem una taula que contindrà les abreviacions que hi hagi al
% nostre document
%
\small{%
\begin{center}
\begin{tabular}{|p{2cm}|p{4.75cm}|p{2cm}|p{4.75cm}|}
\hline
\multicolumn{4}{|>{\columncolor[gray]{.9}[2mm][2mm]}c|}{{\bf Abreviatures}} \\
\hline
UAB & Universitat Autonoma de Barcelona & ES & Enginyeria del Software \\
\hline
& & & \\
\hline
\end{tabular}
\end{center}
}

%
% Insertem una taula amb el historial de canvis del document
%
\small{%
\begin{center}
\begin{tabular}{|p{2cm}|p{2cm}|p{7.5cm}|p{2cm}|}
\hline
\multicolumn{4}{|>{\columncolor[gray]{.9}[2mm][2mm]}c|}{{\bf Historial de Revisions}} \\
\hline
{\bf Version} & {\bf Date} & {\bf Comments} & {\bf Autor} \\
\hline
1.0 & 08-Abril-2019 & Sprint 1 &
G43-2-02 \\
\hline
2.0 & 29-Abril-2019 & Sprint 2 & G43-2-02 \\
\hline
3.0 & 12-Maig-2019 & Sprint 3 & G43-2-02 \\
\hline
4.0 & 26-Maig-2019 & Sprint 4 & G43-2-02 \\
\hline
\end{tabular}
\end{center}
}




% Insertem en una nova pàgina, els indexs : de continguts, de figures
% i de taules, les comandes \tableofcontents ,  \listoffigures,
% \listoftables són gestionades per els paquets fets servir, i per
% defecte escriuen els titols de la secció en angles, per canviar-los,
% hem de 'renovar les comandes' :

\renewcommand{\contentsname}{Índex de continguts}
\renewcommand{\listfigurename}{Índex de figures}
\renewcommand{\listtablename}{Índex de taules}

% Ara li diem al LaTeX que inserti la informació
\newpage
\tableofcontents
\listoffigures
\listoftables

% Insertem les seccions que volem
\newpage

\section{Introducció}\label{sec:intro}

\begin{flushleft}
En este documento se pretende especificar los objetos y funciones de cada una de las clases que formarán parte del programa.\\

De dichas clases se definirán las relaciones entre ellas con tal de poder llevar a cabo todas las interacciones necesarias para el correcto desarrollo del programa.\\
\end{flushleft}

\section{Requeriments Funcionals}\label{sect:RF}


\requeriment{REQ-F-1-01}{USUARIO: Identificación de usuarios}{El usuario se conectará al sistema mediante un usuario y una contraseña.}{A}{R, T}

\requeriment{REQ-F-1-02}{ADMIN: Gestión de mapas}{
El administrador podrá gestionar todos los mapas del juego.}{A}{R, T}

\requeriment{REQ-F-1-03}{ADMIN: Creación de mapas}{
El administrador podrá crear nuevos mapas.}{A}{R, T}{REQ-F-1-02}

\requeriment{REQ-F-1-04}{ADMIN: Modificación de mapas}{
El administrador podrá modificar los mapas ya existentes.}{A}{R, T}{REQ-F-1-02}

\requeriment{REQ-F-1-05}{ADMIN: Eliminación de mapas}{
El administrador podrá eliminar mapas.}{A}{R, T}{REQ-F-1-02}

\requeriment{REQ-F-1-06}{ADMIN: Gestión de cuentas}{
El administrador podrá gestionar las cuentas de los usuarios.}{A}{R, T}

\requeriment{REQ-F-1-07}{ADMIN: Eliminar cuentas}{
El administrador podrá eliminar las cuentas de cualquier usuario.}{A}{R, T}{REQ-F-1-06}

\requeriment{REQ-F-1-08}{ADMIN: Crear cuentas}{
El administrador podrá crear nuevas cuentas para usuarios.}{A}{R, T}{REQ-F-1-06}

\requeriment{REQ-F-1-09}{USUARIO: Gestionar partidas}{
El jugador podrá gestionar sus propias partidas.}{A}{R, T}

\requeriment{REQ-F-1-10}{USUARIO: Empezar partida}{
El jugador podrá iniciar partidas cuando lo desee.}{A}{R}{REQ-F-1-09}

\requeriment{REQ-F-1-11}{USUARIO: Borrar partida}{
El jugador podrá borrar sus partidas cuando lo desee.}{A}{R}{REQ-F-1-09}

\requeriment{REQ-F-1-12}{USUARIO: Continuar partida}{
El usuario puede continuar la partida por donde la dejó.}{A}{R}{REQ-F-1-09}

\requeriment{REQ-F-1-13}{USUARIO: Gestión de partidas existentes}{
El jugador tendrá en el menú la opción para gestionar sus partidas existentes.}{A}{R}{REQ-F-1-09}

\requeriment{REQ-F-1-14}{USUARIO: Nueva partida}{
El jugador tendrá en el menú la opción para empezar una partida nueva.}{A}{R, T}{REQ-F-1-09}

\requeriment{REQ-F-1-15}{ADMIN: Modificar contraseña}{
El administrador podrá modificar la contraseña del usuario.}{A}{R}

\requeriment{REQ-F-1-16}{ADMIN: Modificar ranking}{
El administrador podrá modificar el ranking de los usuarios.}{A}{R}

\requeriment{REQ-F-1-17}{USUARIO: Gestionar los datos de la cuenta}{
El jugador puede gestionar los datos de su propia cuenta.}{A}{R}

\requeriment{REQ-F-1-18}{USUARIO: Modificar datos de la cuenta}{
El usuario puede modificar sus datos de la cuenta.}{A}{R}{REQ-F-1-17}

\requeriment{REQ-F-1-19}{USUARIO: Eliminar datos de la cuenta}{
El usuario puede eliminar sus datos de la cuenta.}{A}{R}{REQ-F-1-17}

\requeriment{REQ-F-1-20}{SISTEMA: Gestionar juego}{
El sistema se encargará de gestionar todo lo relacionado con el juego.}{A}{R}

\requeriment{REQ-F-1-21}{SISTEMA: Gestionar niveles}{
El sistema gestionará los niveles de mayor a menor dificultad.}{A}{R}{REQ-F-1-20}

\requeriment{REQ-F-1-22}{SISTEMA: Gestionar ranking}{
El sistema gestionará el ranking de mayor a menor puntuación.}{A}{R}{REQ-F-1-20}

\requeriment{REQ-F-1-23}{SISTEMA: Gestionar cuentas}{
El sistema gestionará todas las cuentas de usuario (nombre de usuario, contraseña, datos del usuario).}{A}{R}{REQ-F-1-20}

\requeriment{REQ-F-1-24}{SISTEMA: Gestionar mapas}{
El sistema gestionará todos los mapas del juego.}{A}{R}{REQ-F-1-20}

\requeriment{REQ-F-1-25}{ADMIN: Ver estados de los usuarios}{
El administrador podrá ver el estado de los usuarios online (jugando, esperando o desconectado).}{A}{R}

\requeriment{REQ-F-1-26}{USUARIO: Visualización del ranking}{
El jugador tendrá en el menú la opción para ver los rankings (el ranking del propio jugador o el ranking de todos los usuarios del juego).}{A}{R}

\requeriment{REQ-F-1-27}{USUARIO: Modificación de los ajustes}{
El jugador tendrá en el menú la opción para modificar sus ajustes (sonido, vibración, idioma).}{A}{R}

\requeriment{REQ-F-1-28}{USUARIO: Volver al menú principal}{
El jugador tendrá en el menú la opción para volver al menú principal.}{A}{R}

\requeriment{REQ-F-1-29}{USUARIO: Salir del juego}{
El jugador tendrá en el menú la opción para salir del juego.}{A}{R}

\requeriment{REQ-F-1-30}{USUARIO: Guardado manual}{
El jugador podrá guardar la partida manualmente.}{A}{R}

\requeriment{REQ-F-1-31}{ADMIN: Borrar juego}{
El administrador podrá borrar el juego si quiere.}{A}{R}

\requeriment{REQ-F-1-32}{USUARIO: Comenzar partida de nuevo}{
El usuario podrá comenzar partida de nuevo.}{A}{R}
%\section{Requeriments Funcionals}\label{sect:RF}


\requeriment{REQ-F-1-01}{USUARIO: Identificación de usuarios}{El usuario se conectará al sistema mediante un usuario y una contraseña.}{A}{R, T}

\requeriment{REQ-F-1-02}{ADMIN: Gestión de mapas}{
El administrador podrá gestionar todos los mapas del juego.}{A}{R, T}

\requeriment{REQ-F-1-03}{ADMIN: Creación de mapas}{
El administrador podrá crear nuevos mapas.}{A}{R, T}{REQ-F-1-02}

\requeriment{REQ-F-1-04}{ADMIN: Modificación de mapas}{
El administrador podrá modificar los mapas ya existentes.}{A}{R, T}{REQ-F-1-02}

\requeriment{REQ-F-1-05}{ADMIN: Eliminación de mapas}{
El administrador podrá eliminar mapas.}{A}{R, T}{REQ-F-1-02}

\requeriment{REQ-F-1-06}{ADMIN: Gestión de cuentas}{
El administrador podrá gestionar las cuentas de los usuarios.}{A}{R, T}

\requeriment{REQ-F-1-07}{ADMIN: Eliminar cuentas}{
El administrador podrá eliminar las cuentas de cualquier usuario.}{A}{R, T}{REQ-F-1-06}

\requeriment{REQ-F-1-08}{ADMIN: Crear cuentas}{
El administrador podrá crear nuevas cuentas para usuarios.}{A}{R, T}{REQ-F-1-06}

\requeriment{REQ-F-1-09}{USUARIO: Gestionar partidas}{
El jugador podrá gestionar sus propias partidas.}{A}{R, T}

\requeriment{REQ-F-1-10}{USUARIO: Empezar partida}{
El jugador podrá iniciar partidas cuando lo desee.}{A}{R}{REQ-F-1-09}

\requeriment{REQ-F-1-11}{USUARIO: Borrar partida}{
El jugador podrá borrar sus partidas cuando lo desee.}{A}{R}{REQ-F-1-09}

\requeriment{REQ-F-1-12}{USUARIO: Continuar partida}{
El usuario puede continuar la partida por donde la dejó.}{A}{R}{REQ-F-1-09}

\requeriment{REQ-F-1-13}{USUARIO: Gestión de partidas existentes}{
El jugador tendrá en el menú la opción para gestionar sus partidas existentes.}{A}{R}{REQ-F-1-09}

\requeriment{REQ-F-1-14}{USUARIO: Nueva partida}{
El jugador tendrá en el menú la opción para empezar una partida nueva.}{A}{R, T}{REQ-F-1-09}

\requeriment{REQ-F-1-15}{ADMIN: Modificar contraseña}{
El administrador podrá modificar la contraseña del usuario.}{A}{R}

\requeriment{REQ-F-1-16}{ADMIN: Modificar ranking}{
El administrador podrá modificar el ranking de los usuarios.}{A}{R}

\requeriment{REQ-F-1-17}{USUARIO: Gestionar los datos de la cuenta}{
El jugador puede gestionar los datos de su propia cuenta.}{A}{R}

\requeriment{REQ-F-1-18}{USUARIO: Modificar datos de la cuenta}{
El usuario puede modificar sus datos de la cuenta.}{A}{R}{REQ-F-1-17}

\requeriment{REQ-F-1-19}{USUARIO: Eliminar datos de la cuenta}{
El usuario puede eliminar sus datos de la cuenta.}{A}{R}{REQ-F-1-17}

\requeriment{REQ-F-1-20}{SISTEMA: Gestionar juego}{
El sistema se encargará de gestionar todo lo relacionado con el juego.}{A}{R}

\requeriment{REQ-F-1-21}{SISTEMA: Gestionar niveles}{
El sistema gestionará los niveles de mayor a menor dificultad.}{A}{R}{REQ-F-1-20}

\requeriment{REQ-F-1-22}{SISTEMA: Gestionar ranking}{
El sistema gestionará el ranking de mayor a menor puntuación.}{A}{R}{REQ-F-1-20}

\requeriment{REQ-F-1-23}{SISTEMA: Gestionar cuentas}{
El sistema gestionará todas las cuentas de usuario (nombre de usuario, contraseña, datos del usuario).}{A}{R}{REQ-F-1-20}

\requeriment{REQ-F-1-24}{SISTEMA: Gestionar mapas}{
El sistema gestionará todos los mapas del juego.}{A}{R}{REQ-F-1-20}

\requeriment{REQ-F-1-25}{ADMIN: Ver estados de los usuarios}{
El administrador podrá ver el estado de los usuarios online (jugando, esperando o desconectado).}{A}{R}

\requeriment{REQ-F-1-26}{USUARIO: Visualización del ranking}{
El jugador tendrá en el menú la opción para ver los rankings (el ranking del propio jugador o el ranking de todos los usuarios del juego).}{A}{R}

\requeriment{REQ-F-1-27}{USUARIO: Modificación de los ajustes}{
El jugador tendrá en el menú la opción para modificar sus ajustes (sonido, vibración, idioma).}{A}{R}

\requeriment{REQ-F-1-28}{USUARIO: Volver al menú principal}{
El jugador tendrá en el menú la opción para volver al menú principal.}{A}{R}

\requeriment{REQ-F-1-29}{USUARIO: Salir del juego}{
El jugador tendrá en el menú la opción para salir del juego.}{A}{R}

\requeriment{REQ-F-1-30}{USUARIO: Guardado manual}{
El jugador podrá guardar la partida manualmente.}{A}{R}

\requeriment{REQ-F-1-31}{ADMIN: Borrar juego}{
El administrador podrá borrar el juego si quiere.}{A}{R}

\requeriment{REQ-F-1-32}{USUARIO: Comenzar partida de nuevo}{
El usuario podrá comenzar partida de nuevo.}{A}{R}
\section{Requeriments No Funcionals}\label{sect:RNF}


\requeriment{REQ-NF-1-01}{Escalabilidad del juego}{
El diseño debe ser escalable, es decir permitir ampliaciones.}{A}{R}{}

\requeriment{REQ-NF-1-02}{Diseño intutivo}{
El diseño del juego debe ser intuitivo y sencillo de manejar.}{A}{R}

\requeriment{REQ-NF-1-03}{Rendimiento}{
El sistema debe soportar como mínimo 300 jugadores.}{A}{R}

\requeriment{REQ-NF-1-04}{
Diseño adaptado a navegador web}{El diseño debe estar programado de forma que funcione desde un navegador.}{A}{R}

\requeriment{REQ-NF-1-05}{Seguridad}{
El sistema debe ser seguro (cifrados de contraseñas).}{A}{R}

\requeriment{REQ-NF-1-06}{Copias de seguridad}{
El servidor realizará copias de seguridad cada cierto tiempo para garantizar que no se pierden datos.}{A}{R}


\section{Traçabilitat}\label{sec:intro}

\begin{center}
\begin{tabular}{|c|c|c|}
\hline
{\cellcolor[gray]{.8} \bf Requeriment} & {\cellcolor[gray]{.8} \bf Cas d'ús} & {\cellcolor[gray]{.8} \bf Test}  \\
\hline
REQ-F-1-01 & CU-USUARI-01 & TP-U-02 \\
\hline
REQ-F-1-02 & - & - \\
\hline
REQ-F-1-03 & CU-ADMIN-02 & - \\
\hline
REQ-F-1-04 & CU-ADMIN-03 & - \\
\hline
REQ-F-1-05 & CU-ADMIN-04 & - \\
\hline
REQ-F-1-06 & - & - \\
\hline
REQ-F-1-07 & CU-ADMIN-05 & TP-U-03 \\
\hline
REQ-F-1-08 & CU-ADMIN-06 & TP-U-01 \\
\hline
REQ-F-1-09 & - & - \\
\hline
REQ-F-1-10 & CU-USUARI-07 & - \\
\hline
REQ-F-1-11 & CU-USUARI-08 & TP-U-04 \\
\hline
REQ-F-1-12 & CU-USUARI-09 & - \\
\hline
REQ-F-1-13 & CU-USUARI-10 & - \\
\hline
REQ-F-1-14 & CU-USUARI-11 & - \\
\hline
REQ-F-1-15 & CU-ADMIN-12 & TP-U-05 \\
\hline
REQ-F-1-16 & CU-ADMIN-13 & - \\
\hline
REQ-F-1-17 & - & - \\
\hline
REQ-F-1-18 & CU-USUARI-14 & - \\
\hline
REQ-F-1-19 & CU-USUARI-15 & - \\
\hline
REQ-F-1-20 & - & - \\
\hline
REQ-F-1-21 & CU-SISTEMA-16 & - \\
\hline
REQ-F-1-22 & CU-SISTEMA-17 & - \\
\hline
REQ-F-1-23 & CU-SISTEMA-18 & - \\
\hline
REQ-F-1-24 & CU-SISTEMA-19 & - \\
\hline
REQ-F-1-25 & CU-ADMIN-20 & TP-U-06 \\
\hline
REQ-F-1-26 & CU-USUARI-21 & - \\
\hline
REQ-F-1-27 & CU-USUARI-22 & - \\
\hline
REQ-F-1-28 & CU-USUARI-23 & - \\
\hline
REQ-F-1-29 & CU-USUARI-24 & - \\
\hline
REQ-F-1-30 & CU-USUARI-25 & TP-U-07 \\
\hline
REQ-F-1-31 & CU-ADMIN-26 & - \\
\hline
REQ-F-1-32 & CU-USUARI-27 & - \\
\hline

\end{tabular}
\captionof{table}{Matriu Traçabilitat}
\end{center}


% Finalitzem el document.
\end{document}
